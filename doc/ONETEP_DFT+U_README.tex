\documentclass[a4paper,oneside,11pt]{article}

\usepackage{graphicx}
\usepackage{setspace}
\usepackage{latexsym,amsmath,amssymb,amsthm}

\newenvironment{alist}
{\renewcommand{\theenumi}{\alph{enumi}}
\renewcommand{\labelenumi}{\textup{(\theenumi)}}
\renewcommand{\theenumii}{\roman{enumii}}
\renewcommand{\labelenumii}{\textup{(\theenumii)}}

\begin{enumerate}}
{\end{enumerate}}

\newenvironment{rlist}
{\renewcommand{\theenumi}{\roman{enumi}}
\renewcommand{\labelenumi}{\textup{(\theenumi)}}
\begin{enumerate}}
{\end{enumerate}}

\newcommand{\dummy}[1]{\newcounter{#1}\setcounter{#1}{\value{enumii}}}
\newcommand{\rref}[1]{\tu{(\roman{#1})}}

\setcounter{secnumdepth}{2}
\setcounter{section}{1}

\numberwithin{equation}{section}

\usepackage{fancyhdr}
\pagestyle{fancy}
\rhead{}

\pagenumbering{roman} \setcounter{page}{1}

\begin{document}
\title{Notes for users of the DFT+U (a.k.a. LDA+U) implementation in ONETEP}
\author{David D. O'Regan \\  University of Cambridge}
\date{August 2009}

\maketitle

\section*{WHAT IS DFT+U?}

The DFT+U method \cite{Anisimov91,Anisimov97, Dudarev} is used to improve upon the description
of so-called strongly correlated materials offered by DFT within  
approximations for exchange-correlation such as LSDA, $\sigma$-GGA etc.
These functionals based on the locally-evaluated density can 
often fail to reproduce the physics associated with localised
orbitals of $3d$ and $4f$ hydrogenic character found in strongly 
correlated materials, a category traditionally consisting of first-row 
transition metals and their oxides but increasingly being considered to include
Lanthanoid oxide materials, magnetic semiconductors and
organometallic molecules. 

Typically, the LDA and extensions thereof
can underestimate both local moments on metal ions and the insulating gap. Correlation gap underestimation 
(due to neglect in the LDA of the derivative discontinuity 
with respect to orbital occupancy in the exact XC-funtional)
may be confounded by an underestimation of the exchange splitting
(the exchange gap can best be recoved by using a functional including
long-range exchange e.g. based partly or entirely on the Hartree-Fock approximation).

The DFT+U correction term is usually thought of
as a treatment of the exchange-correlation energy contributed 
by the correlated sites (subspaces projected out with functions 
of $3d$ and or $4f$ character on ``Hubbard atoms") within the Hubbard 
model, including a double-counting correction for that contribution
already included in the LDA term. The flavour implemented in 
ONETEP is the basis-set independent, rotationally invariant version
of reference \cite{Cococcioni1}, where nonsphericity in the electronic interaction
and explicit exchange coupling are neglected, leading
to the DFT+U correction term

\begin{equation}
E_U \left[ n^{(I) (\sigma)} \right] =  \sum_{I \sigma} \frac{U^{(I)}-J^{(I)}}{2} Tr \left[  n^{(I) (\sigma)} \left( 1 -  n^{(I) (\sigma)} \right)\right],
\end{equation}

where $U$ and $J$ are the screened Hubbard on-site repulsion and exchange interaction parameters and the occupancy matrix of the correlated site $I$ for spin channel $\sigma$ is defined by

\begin{equation}
n^{(I)(\sigma)}_{m m'} = \sum_i f_i \langle \psi_i^{(\sigma)} \rvert \hat{P}^{(I)}_{m m'} \lvert \psi_i^{(\sigma)} \rangle,
\end{equation} 
defining $\hat{P}^{(I)} = \lvert \varphi^{(I)}_m \rangle \langle \varphi^{(I)}_{m'} \rvert$ in the case of a set of
orthogonal Hubbard projectors localised on site $(I)$.

Put simply, if the system under study contains first-row transition-metal 
or Lanthanoid ions there is a good chance that the LDA will partly 
occupy degenerate $3d$ or $4f$ states rather than splitting them into occupied and
virtual Hubbard bands, hence underestimating the correlation gap and 
associated magnetic order. In this case the DFT+U method can be used to 
penalise the noninteger occupancy of these states, tending to fill states
 with occupancy greater than $0.5$ and emptying states with occupancy less than $0.5$, 
as can be seen from the expression for the DFT+U potential
\begin{equation}
\hat{V}^{(\sigma)}_{DFT+U} = \sum_{(I)} \left( U^{(I)}-J^{(I)} \right) Tr \left[ \frac{1}{2} - n^{(I) (\sigma)} \right].
\end{equation} 

The DFT+U term may 
be considered to be a correction which removes the contribution to the energy
arising due to the spurious self-interaction of a partially occupied
orbital \cite{Cococcioni1}. In this case, the $U-J$ parameter (henceforth simply denoted $U$)
is the curvature of the total energy with respect to the occupancy
of the correlated manifold - which should be a piece-wise linear curve
were it to satisfy Janak's theorem \cite{Janak} - and can be computed using linear-response
theory (among other methods such as constrained DFT) according to the prescription given in references \cite{Cococcioni1,Cococcioni2}.

\section*{HOW TO USE DFT+U IN ONETEP}

In order to activate the DFT+U functionality, the \textbf{hubbard} block
is used in the input file. For example, in the case of a system 
containing Iron and Cerium atoms, which we think needs the 
help of the DFT+U correction to improve the description of localisation, 
we might use the hubbard block 

\begin{table}[!h]
{\centering \begin{tabular}{|c|c|c|c|c|c|}
\hline\hline
!! species$^1$      &  $l$ $^2$ & $U$(eV)$^3$ & Z$^4$ & $\alpha$(eV)$^5$  & $\sigma$-splitting(eV)$^6$   \\
\hline
\% block hubbard & & & & & \\
Fe1 & 2 & 4.0 & 11.18 & 0.00 & 0.05    \\
Fe2 & 2 & 4.0 & 11.18 &  0.00 & -0.05    \\
Ce1 & 3 & 6.0  & 1.68 & 0.05 & 0.0     \\
\% endblock hubbard & & & & & \\
\hline\hline
\end{tabular}\par}
\end{table}

The columns of the \textbf{hubbard} block are described as follows:

\begin{enumerate}
\item The species label, e.g. $Fe1$ for Iron atoms of the 1st type in the cell, $Fe2$ for Iron atoms of a 2nd kind etc.

\item The angular momentum channel of the projectors
used to delineate the strongly correlated sites on Hubbard atoms of type, e.g. $l=2$ for $Fe1$. 
The radial quantum number $r=l+1$ is used to generate atom-centred hydrogenic projectors, so $l=2$ gives $3d$ orbitals, $l=3$ gives $4f$ orbitals etc.

\item The value of the Hubbard $U$ to use, in electron-volts. 
Most users will simply work with the value for $U$ which corrects the
band-gap or bond-lengths in the system they wish to study. Methods 
do, however, exist to estimate its value, for example the linear-response technique \cite{Cococcioni1, Cococcioni2} which is 
implemented in ONETEP. 

\item The effective charge divided by the ratio of effective masses 
is needed in order to generate the hydrogenic Hubbard projectors. It has been 
shown that this does make a difference to the effect of the DFT+U method \cite{Pickett}, but 
some feel that what is most important is to choose a value for $U$ which is
consistent with the choice of projectors \cite{Cococcioni1}. Valence pseudo-orbitals from the
pseudopotential generator may be a preferable choice, or Wannier functions
such as those used in the self-consistent scheme described below. 
In this example we have used the Clementi-Raimondi \cite{Clementi1,Clementi2} effective charges for 
neutral Iron and Cerium, $3d$ and $4f$ orbitals respectively, this data is quite
old, and based on the restricted Hartree-Fock method, but serves as a useful
guess. The hydrogenic projectors are renormalised within an atom-centred sphere
with the same radius as the NGWFs on that atom.

\item An on-site potential acting on the Hubbard projectors, 
the prefactor $\alpha$ is here entered in electron-volts. This is needed in 
order to vary the occupancy of the correlated subspace in order to determine
the value of $U$ which is consistent with the screened response in the system
with linear-response theory \cite{Cococcioni1, Cococcioni2}. In this case we seem to be trying to calculate a
value of $U$ for the Cerium atoms.

\item The spin-splitting factor, in electron-volts, which is
deducted from the $\alpha$ factor for the spin-up channel and added
to $\alpha$ for the spin-down channel. In this case it looks like we're 
encouraging occupation of the spin-up channel for some Iron atoms, and 
spin-down for others. Perhaps this is an antiferromagnetically ordered
oxide of Iron which is doped with Cerium, and we're worried about maintaining
this ordering while we deduce the $U$ parameter for Cerium.

\textbf{N.B.} Some users may find the above described functionality useful in 
cases of spin or charge ordered systems even when the DFT+U correction
is not needed. Moreover, with zero $U$, $\alpha$ and spin-splitting 
factors, the DFT+U functionality may be used to monitor the projection
of the density matrix onto a set of hydrogenic projectors of your choosing.

In the case of antiferromagnetically ordered solids,
it is very unlikely that any currently available DFT code will find 
the correct ground state from an un-spin-ordered starting guess. 
The user may find it useful to use a non-zero 
spin-splitting factor on atoms which are known to have a given magnetic moment for a few NGWF iterations before restarting in order 
to relax to the ground state.
Similary, it may be found difficult to locate a charge-ordered ground state
in cases where there are no geometric grounds for charge to locate preferentially
in one part of a system, leading to partial occupancies even when DFT+U is
used. The alpha parameter may be set nonzero temporarily in this case with a zero value
of $U$ in order to find the charge-ordered state even in cases where DFT+U is
not needed.
\end{enumerate}

\textbf{N.B.} At present the DFT+U contribution to the forces is not implemented
and thus molecular dynamics and geometry optimisation tasks are incompatable
with the DFT+U functionality.

\section*{MORE ADVANCED OPTIONS}

Any set of localised functions may, in principle, be used for Hubbard projectors since the
choice is always somewhat arbitrary. However, it is becoming increasingly accepted
that Wannier functions, in particular the Maximally Localised variety \cite{Marzari, Souza}
can provide an excellent minimal basis with which to construct tight-binding models from ab 
initio simulations. These have been used with good effect in the past to construct sophisticated
self-interaction corrections based on the occupancies of localised projectors \cite{Mazurenko, Lechermann}.
Indeed, there is numerical evidence to suggest that Maximally Localised Wannier Functions provide
the basis which maximises the corresponding on-site repulsion, and hence the $U$ parameter \cite{Miyake}.

In ONETEP, the NGWFs are a readily accessible set of localised Wannier functions, related to the 
MLWFs via the Modern Theory of Polarisation. Thus, in ONETEP it is possible to re-use the NGWFs from
the end of a ground-state calculation as a set of Hubbard projectors with which to construct the DFT+U correction. 

In order to ensure that NGWFs with the correct symmetry are chosen as Hubbard projectors on a given
atom, those $l$ NGWFs $\phi$ centered on that atom which maximise $\sum_l \lvert \langle \phi \rvert \varphi_l\rangle \rvert^2$,
for the set of hydrogenic projectors $\varphi_l$ defined in the \textbf{hubbard} block, are selected for the task.

The keyword \textbf{hubbard\_max\_iter}, (default of $0$), sets the task to \textbf{HUBBARDSCF}
and carries out a self-consistency loop over the Hubbard projectors. The density from one minimisation is re-used at the beginning of the next,
and setting \textbf{hubbard\_max\_iter} to $2$ one can carry out one DFT+U calculation on the back of 
a conventional ONETEP minimisation. 

The keywords \textbf{hubbard\_energy\_tol} , \textbf{hubbard\_conv\_win} and  \textbf{hubbard\_proj\_mixing} are used
to obtain full self-consistency over the Hubbard projectors for a given $U$ parameter. The ground state energy must
deviate less than \textbf{hubbard\_energy\_tol} (default of $10^{-6}Ha$) from one  \textbf{HUBBARDSCF} iteration 
to the next over  \textbf{hubbard\_conv\_win} (default of $2$) iterations. A fraction  \textbf{hubbard\_proj\_mixing} (default of $0.0$) 
of the previous Hubbard projectors may be mixed with the new ones in order to prevent projector oscillation, though
it must be stressed that the energy is not guaranteed to decrease monotonically with successive \textbf{HUBBARDSCF} iterations.

\begin{thebibliography}{13}

\bibitem{Anisimov91}
J. Z. V. I. Anisimov and O. K. Andersen, 
	Phys. Rev. B \textbf{44}, 943 (1991).

\bibitem{Anisimov97}
V. I. Anisimov, F. Aryasetiawan, and A. I. Liechtenstein, 
	J. Phys.: Condens. Matter \textbf{9}, 767 (1997).

\bibitem{Janak}
J. F. Janak,
Phys. Rev. B \textbf{18}, 12 (1978).

\bibitem{Lechermann}
F. Lechermann, A. Georges, A. Poteryaev, S. Biermann, M. Posternak, A. Yamasaki and O. K. Andersen,
Phys. Rev. B \textbf{74}, 125120 (2006).

\bibitem{Marzari}
N. Marzari and D. Vanderbilt,
Phys. Rev. B \textbf{56}, 12847 (1997).

\bibitem{Souza}
I. Souza, N. Marzari and D. Vanderbilt,
Phys. Rev. B \textbf{65}, 035109 (2001).

\bibitem{Miyake}
T. Miyake and F. Aryasetiawan,
Phys. Rev. B \textbf{77}, 085122 (2008).

\bibitem{Cococcioni1}
M. Cococcioni and S. de Gironcoli,
Phys. Rev. B \textbf{71}, 035105 (2005).

\bibitem{Cococcioni2}
H. J. Kulik, M. Cococcioni, D. A. Scherlis and N. Marzari,
Phys. Rev. Lett. \textbf{97}, 103001 (2006).

\bibitem{Mazurenko}
V. V. Mazurenko, S. L. Skornyakov, A. V. Kozhevnikov, F. Mila and V. I. Anisimov,
Phys. Rev. B \textbf{75}, 224408 (2007).

\bibitem{Dudarev}
S. L. Dudarev,
Phys. Rev. B \textbf{57}, 3 (1998).

\bibitem{Pickett}
W. E. Pickett, S. C. Erwin and E. C. Ethridge,
Phys. Rev. B \textbf{58}, 1201 (1998).

\bibitem{Clementi1}
E. Clementi and D.L.Raimondi,
 J. Chem. Phys. \textbf{38}, 2686 (1963). 

\bibitem{Clementi2}
E. Clementi, D.L.Raimondi, and W.P. Reinhardt,
 J. Chem. Phys. \textbf{47}, 1300 (1967).

\end{thebibliography}

%\newpage

\addcontentsline{toc}{section}{References}
\bibliographystyle{unsrt}

\end{document}
